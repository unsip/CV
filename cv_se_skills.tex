\documentclass[10pt]{report}
\usepackage{enumitem}
\usepackage{float}
\usepackage[margin=30pt]{geometry}
\usepackage{hyperref}
\usepackage{unicode}

\pagestyle{empty}
\hypersetup{
    colorlinks=true,
    urlcolor=blue,
}
\setlist[description]{leftmargin=*}
\setlist[itemize]{leftmargin=*}

% Job listing
\newenvironment{JobDescription}[5]{
    \vspace{ #5 }
    \flushleft
    {\bf #1 } \hfill { #2 }
    \\
    {\em #3 } \hfill {\em #4 }
    \begin{itemize}
} {
    \end{itemize}
}

\begin{document}
\begin{tabular}{@{}p{.60\textwidth}rp{.35\textwidth}}
    \bf{\LARGE{Andrey Turbov} \newline{\small{Sep 5, 1996}}} & & {\bf Contact} \\
    & {\small LinkedIn:}    & {\small \href{https://linkedin.com/in/andrey-turbov-8a6a91196}{Andrey Turbov}} \\
    & {\small E-mail:}      & {\small unsip@tuta.io} \\
    & {\small Phone:}       & {\small +995 591812617} \\
    & {\small Github:}      & {\small \href{https://github.com/unsip}{https://github.com/unsip}}
\end{tabular}

\vspace{5mm}
{\noindent
    I'm a software and operations engineer with a few years of SRE expertise. Self-education is of the highest value for
    me, which I'm always striving. Not only because regular studies got me into the industry, but also because I
    appreciate scientific endeavour and enjoy exploring world of abstract ideas. Hence, I'm looking forward to further
    improve my skills in software development, architecture design and DevOps practices. The essential job-aspect for me
    is to acquire expertise in solving challenging problems using modern technologies and cutting edge approaches. My
    experience in an international environment enables me to get along with foreign co-workers and customers quickly.
}

\subsection*{Employment}
\begin{JobDescription}{Freelance}{Jul 2023 --- Present}{Software Engineer}{Tbilisi}{2mm}
    \item Developed initial platform prototype for automatic conversion/generation of \emph{CMake} files with
        \emph{Django/DRF}
    \item Implemented new platform features according to customer requirements regarding their build system improvements
    \item \emph{Django/DRF/ReactJS} service for automated crypto infrastructure deployment to various cloud providers:
    \begin{itemize}
        \item Implementation of new features and bug reduction via legacy code refactoring
        \item Introduced TDD and existing code coverage with \emph{pytest} + CI/CD pipelines configuration
        \item Monitoring alerts configuration catching critical application issues such as stale deployments, queues
            overflow, etc.
    \end{itemize}
\end{JobDescription}

\begin{JobDescription}{JettyCloud (RingCentral)}{Dec 2022 --- Apr 2023}{Release Engineer}{Tbilisi}{2mm}
    \item Design and maintain internal custom \emph{Python} toolchain for release routine such as mass services
        deployment
    \item Building and decommission of instances depending on a business needs
    \item Carry out release preparation cycle including:
    \begin{itemize}
        \item Deployment of the whole RC distributed system
        \item Tracing and resolving issues which occured during QA regression cycles
        \item Bug reporting and discussion with developers on possible solutions
    \end{itemize}
\end{JobDescription}

\begin{JobDescription}{}{Apr 2022 --- Dec 2022}{Site Reliability Engineer}{Tbilisi}{-3mm}
    \item Deployment and configuration on production environments in a strict time frames
    \item Subsystems support handover from other teams with complex review and presentation for peers
    \item Tracing and resolving critical application issues in a complex production environment
    \item Tight communication with international customers, operations and development teams to achieve system reliability
    \item Deployment processes automation with \emph{Python} scripts
\end{JobDescription}

\begin{JobDescription}{Teknavo (Bloomberg)}{Dec 2021 --- Apr 2022}{DevOps Engineer}{St. Petersburg}{5mm}
    \item Internal Bloomberg CI/CD infrastructure improvement to speed up build and release process
    \item Build and testing automation with \emph{Python}, \emph{CMake}, \emph{GTest} and \emph{pytest}
    \item \emph{C++} codebase tests coverage and bugfixing
\end{JobDescription}

\begin{JobDescription}{T-Systems (Deutsche Telekom)}{Jan 2020 --- Dec 2021}{Software Developer}{St. Petersburg}{0mm}
    \item Design new system architecture for \emph{Python Django} RPA platform with UML
    \item Refactoring of a legacy \emph{Django/DRF/Celery/Flask} backend according to new system design
    \item Transition from RPC (\emph{RPyC}) protocol to plain REST API to simplify backend-agent communication
    \item Implemented multiprocessing \emph{Python} RPA agent-side for easy bots management and concurrent execution
    \item \emph{ReactJS} frontend development and improvements following strict company UI standards
    \item Introduced TDD practice for both frontend and backend:
    \begin{itemize}
        \item Frontend unit-test configuration and coverage using \emph{Jest} and \emph{Enzyme}
        \item Backend and agent unit-testing with \emph{pytest}, \emph{factory\_boy} for data mocking, \emph{pytest-cov}
            for coverage metrics
        \item Organized knowledge sharing sessions on how to properly unit-test components using \emph{pytest}
        \item Paired programming sessions with colleagues to help with TDD adoption
    \end{itemize}
\end{JobDescription}

\begin{JobDescription}{}{Jul 2018 --- Jan 2020}{DevOps Engineer}{St. Petersburg}{-3mm}
    \item Monthly presentations on international on-site meetings
    \item Support for international DEV, QA and SRE teams
    \item Orchestration solution design and development using \emph{Puppet} and \emph{Puppet Bolt}
    \item Migration of CI/CD pipelines from \emph{Jenkins} to \emph{GitLab CI}
    \item Implementation and maintainance of internal services and utilities such as:
    \begin{itemize}
        \item \emph{Django/DRF/ReactJS} dashboard web service for project environments \item \emph{Flask} service for generating delivery instructions for production OPS \item \emph{Python} CLI utility for complex application configuration management
    \end{itemize}
\end{JobDescription}

\subsection*{Education}
\begin{table}[H]
    \begin{tabular}{@{}p{0.15\textwidth}p{0.90\textwidth}}
        2014 --- 2016 & Saint Petersburg State University of Aerospace Instrumentation, Department of Computer Science \\
        2013 --- 2014 & Peter the Great Saint-Petersburg Polytechnic University, Department of Physics and Astronomy
    \end{tabular}
\end{table}

\subsection*{Used Tools}
\begin{description}
    \item[Programming languages:]
    Python, C++(11/14/17), JavaScript (ES6), Elixir, Groovy
    \item[Frameworks:]
    Django, Django REST Framework (DRF), Flask, Celery, RPyC, pytest, ReactJS, Jest/Enzyme, GTest
    \item[Development utilities:]
    Git, Bash, core *nix utilities, rpm, dpkg, Sphinx
    \item[Operating systems:]
    Linux (RHEL, Debian, Arch, Gentoo, Exherbo), Windows
    \item[Build systems and CI/CD:]
    Make, Ninja, CMake, Gitlab CI, Jenkins
    \item[Containers/Cloud/Virtualization:]
    Docker, k8s/Spinnaker, AWS, Vagrant, VirtualBox, VMware
    \item[Infrastructure orchestration:]
    Ansible, Puppet, Puppet Bolt, Terraform
    \item[Monitoring:]
    ELK stack, Grafana, Prometheus, Zabbix
    \item[Databases:]
    PostgreSQL, OracleDB, Amazon RDS
\end{description}

\subsection*{Other}
\begin{description}
    \item[Languages:]
    advanced English, native Russian
    \item[Achievements:]
    participated in student physics and astronomy conferences/competitions,
    completed \href{https://gsl.mit.edu/}{MIT GSL} program
    \item[Hobbies:]
    \href{https://github.com/unsip}{programming}, \href{https://exherbo.org/}{Exherbo Linux} distribution maintainance,
        philosophy and humanity studies, digital art, spider keeping
\end{description}

\end{document}
